
% Author: PokMan Ho pok.ho19@imperial.ac.uk
% Script: LogisticTmp.tex
% Desc: `LaTex` report framework -- Logistic Growth
% Input: none
% Output: none
% Arguments: 0
% Date: Oct 2019

\documentclass[a4paper, 11pt]{article}
\usepackage[margin=1in]{geometry}
\usepackage{hyperref, setspace, lineno}

%% test insert graphs
\usepackage{graphicx}
\graphicspath{ {../results/} } %% <https://www.overleaf.com/learn/latex/Inserting_Images>

%% test insert variables
\newcommand{\ReportTitle}{Phenological models on microbial growth -- which model is better and why?} %% <https://stackoverflow.com/questions/1211888/is-there-any-way-i-can-define-a-variable-in-latex>
\newcommand{\ReportAuthor}{PokMan HO}
\newcommand{\ReportAffil}{Department of Life Sciences, Faculty of Natural Sciences,\\Imperial College London}
\newcommand{\Disclaim}{\textbf{A Mini-project submitted in partial fulfilment of the requirements for the degree of Master of Research at Imperial College London\\\\Formatted in the journal style of the \textit{Nature} Journal\\Submitted for the MRes in Computational Methods of Ecology and Evolution}}

\title{\ReportTitle}
\author{\ReportAuthor (CID: 01786076)}
\date{}

%% citation
\usepackage[%
autocite    = superscript,
backend     = bibtex,
sortcites   = true,
style       = nature
]{biblatex}
\bibliography{../reference/LogRef.bib} %% <https://tex.stackexchange.com/questions/6805/bib-library-file-in-a-different-directory-how-to-use-mendeley-centralised-b>

%% set as required
\doublespacing
\linenumbers

\begin{document}
	\begin{center}
		\Huge\textbf{\ReportTitle}\\
		\LARGE\ReportAuthor\\
		\Large\ReportAffil
	\end{center}
	\begin{figure}[h]
		\centering\includegraphics[width=\linewidth]{icl.jpg}
	\end{figure}
	\begin{flushright}
		\Large Approximate Word Count: %% insert approx word count
	\end{flushright}
	\clearpage
	
	\maketitle
	\section*{Abstract}
	
	
	\section*{Introduction}
	Phenological models are expected to fit data trends within its biological field.  Yet due to different reasons, models developed and published from one sample may not fit the others.  These reasons may be due to data variabilities, confounding factors, inaccurate assumptions or models being too-specific.  This project is aimed at compare and contrast different published phenological models different microbial population growth data, which is a better one under what conditions.  The hypotheses are:
	\begin{itemize}
		\item published phenological models are better than polynomials in describing microbial population data;
		\item appropriate phenological model(s) can be identified through distinguishable shapes of microbial population data; and
		\item parameters of data under each model is clustered, similar with dataset best-described by the same model but different from those described by other models.
	\end{itemize}
	
	\section*{Methods}
	
	
	\subsection*{Computing tools}
	R (ver 3.6.0)\autocite{Rcore} was used with following packages: ``ggplot2"\autocite{ggplot2} was used for visualisation; ``reshape2"\autocite{reshape2} was used for converting dataset from wide to long format; ``scales"\autocite{scales} was used for improve ``ggplot" graphs data presentation; and ``minpack.lm"\autocite{minpacklm} was used for computing non-linear least square statistics for model comparisons.
	
	\section*{Results}
	
	
	\section*{Discussion}
	%% AIC vs BIC <https://www.methodology.psu.edu/resources/AIC-vs-BIC/>
	%% AIC <https://en.wikipedia.org/wiki/Akaike_information_criterion#Comparison_with_BIC>
	%% BIC <https://en.wikipedia.org/wiki/Bayesian_information_criterion>
	Model fitness to real data and simplistic mathematics were favoured by both AIC\autocite{johnson2004model,akaike1998information,burnhamdr} and BIC\autocite{johnson2004model,turchin2003complex}.  Apart from that, BIC also takes account of sample size effect\autocite{johnson2004model,turchin2003complex}.\\
	comparisons in different fields\autocite{kuha2004aic,aho2014model,yang2005can,vrieze2012model,wang2006comparison,acquah2010comparison}
	
	\section*{Conclusion}
	
	\section*{Code and Data Availability}
	All \href{https://github.com/ph-u/CMEECourseWork_pmH/tree/master/MiniProject/code}{scripts} and \href{https://github.com/ph-u/CMEECourseWork_pmH/tree/master/MiniProject/data}{data} used for this report were publicity available at GitHub.
	\nocite{*}\printbibliography
\end{document}
