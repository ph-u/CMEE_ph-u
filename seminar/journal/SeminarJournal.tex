\documentclass{article}
\usepackage[utf8]{inputenc}
\usepackage[margin=1.3in]{geometry}
\usepackage{fancyhdr,graphicx}
\usepackage[%
autocite    = superscript,
backend     = bibtex,
sortcites   = true,
style       = nature,
]{biblatex}

\title{Imperial College London\\2019-20 MRes CMEE Seminar Summary}
\author{PokMan Ho (01786076)}
\date{June 2020}

\pagestyle{fancy}
\fancyhead[RE,LO]{Seminar Journal}
\fancyhead[C]{2019-2020 ICL CMEE MRes}
\fancyhead[RO, LE]{CID: 01786076}
\lfoot{\includegraphics[width=.3\headwidth]{icl.png}}
\fancyfoot[C]{\thepage}
\fancyfoot[R]{PokMan HO}

\begin{document}

\maketitle
\tableofcontents
\begin{figure}
    \centering
    \includegraphics[width=.5\linewidth]{icl.png}
\end{figure}
\clearpage
    \section{Deep-time evolution of biological responses to temperature changes}
    \begin{flushright}
        \large{Dr. Dimitrios-Georgios Kontopoulos (Imperial College London)}\\
        seminar date: 10-Oct-2019
    \end{flushright}
    Climate change is logically one of the major threats to modern biodiversity.  This project hence is providing a baseline information of how fast evolution is coping with temperature changes.  A variant of Sharpe-Schoolfield model was used to describe the Thermal Performance curve (TPC) of a population.  The model had an assumption of ``an unit's growth rate considered in the model is only depended on one rate-limited thermal-driven enzyme".\\\\
    Unsurprisingly, most of the results were expected.  Thermodynamic constraints caused a negative correlation between thermal optimal performance (B$_{pk}$) and the temperature difference between the half-to-peak efficiency (W$_{op}$).  Cell size was weakly- and negatively-correlated with enzymatic peak performance.  Existing species were having diversity around local thermal optimums.  Phylogenetically patchy evo-rate shifts on thermal sensitivity with high environmental factors involvement.  Protein mutations had higher damages in high temperatures due to the increase of sub-molecular kinetics.\\\\
    There was a point which lacked further explanation.  ``High temperature select for lower substitution rate" (a result from presentation) does not firmly oppose the idea of ``high temperature favours evolution" (the metabolic theory) in this presentation.  High temperature increase substitution probabilities in all single nucleotide positions (SNPs) can also accommodate both statements.  So even though most of the mutations were selected against (due to the destructive protein kinetics), the rate of getting beneficial mutations was still higher than other scenarios.  So more evidence would be beneficial to completely falsify the theory.\\\\
    In conclusion, the presentation only gave numerical proof to existing knowledge.  Hence, more investigations should be done in the future to put this in context of the current climate change.
    \clearpage
    
    \section{A manifesto for systematically describing consumer-resource interactions}
    \begin{flushright}
        \large{Dr. Daniel Barrios-O'Neill (University of Exeter)}\\
        seminar date: 31-Oct-2019
    \end{flushright}
    Organisms' fitness are tightly bonded with their ability to survive.  Resource consumption is hence a crucial ability to master.  However phenotypic plasticity is a extremely-complex stimuli-response mechanism that is difficult to breakdown.  This seminar has a well-thought out plan to design appropriate methodologies for these pairwise interactions.\\\\
    It is hard to seek for a general equation for describing these important interactions because of two main reasons -- the existing data is taxonomically-skewed and there are a high number of pseudo-replications among published data.  Yet, large endotherms were still found to have statistically higher capture rate and lower handling time than the others.\\\\
    3-D printing was used to try isolating pairwise consumer-resource interactions for active predators.  Good quality data was generated from the methodological precision designed to minimize co-variation effects.  However it is inevitably hard to technically control refuge space without customizing the artificial habitat features with ratio based upon the predator body size.\\\\
    A problem of contemporary bio-data collection was well-highlighted -- the inconsistency and insufficiency of data attributes.  This effect was magnified for poorly-represented taxa clades, including passive predators and oceanic benthic organisms.  In this seminar, the impact of missing data attributes were very obvious because this problem has made data unavailable for statistical analyses.  Hence data attribute standardization should be seriously enforced in order to make the overflowing data being useful to data-mining researches.\\\\
    In conclusion, this seminar was inspirational on designing methodologies and collecting bio-data.
    \clearpage
    
    \section{Climate-driven variation in mosquito density predicts the spatiotemporal dynamics of dengue}
    \begin{flushright}
        \large{Dr. Ruiyun Li (Imperial College London)}\\
        seminar date: 14-Nov-2019
    \end{flushright}
    Dengue fever is a serious threat to coastal populations.  Under climate change impact, this climate-sensitive mosquito-borne disease is expanding towards the inland.  Currently about half of the global population is at risk.  Through a decade-long mosquito population and dengue case data in China, the presenter has combined different drivers and risk factors into a simple-parameterized integrated model.\\\\
    Two factors were investigated -- ``mosquito abundance" and ``per mosquito vector efficiency".  Two species of mosquito data were combined because the difference between species were not biologically apparent.  Five southern cities were included in the field sampling, which has several severe dengue outbreaks within the latest decade.  Although human exposure time can be a potential confounding factor, it was not considered in the primitive model.\\\\
    Mosquito abundance was found to be dominating the outbreak patterns and the simple model was able to capture most of the past severe outbreaks in time and severity.  The result has indicated the current approach of mosquito control is correct.  Since there was an observed one-month time lag between climate pattern changes and mosquito abundance, an accurate model can potentially be an effective alarm for the health departments to react and implement strategic plans against potential dengue outbreaks in the future.\\\\
    In the presentation, the presenter has talked about constructing a susceptible-infected-recovered (SIR) model as a proceeding predictive model to the above one.  However a more complex model require more high-quality data for calibration.  In conclusion, the current model is powerful in prediction of climate-driven local dengue outbreaks.  Yet if the associated governmental bodies are responding accordingly, dengue can potentially be controlled and there would not be enough data for the SIR model calibration.
    \clearpage
    
    \section{Flowers, bees and shifting seasons: how do adapt when Nature's calendar goes out of sync in a warming world}
    \begin{flushright}
        \large{Dr. Jacob Johansson (Lund University, Sweden; Imperial College London)}\\
        seminar date: 21-Nov-2019
    \end{flushright}
    Under the warming climate trend, it is unclear that whether fitness of bees would be affected by shifting flowering season.  These phenological mismatches are potentially leading to negative fitness effect but demographic factor was seldom considered.  In this research, general graphic shape functions were emphasized.\\\\
    Four models were used to investigate the optimal switching time between somatic and reproductive part of a hive -- the time of producing workers and sexuals.  These phenological models are 1. seasonal variation in production rate; 2. demographic responses; 3. resource competition; and 4. interspecific competition.  Through different models, it is hopefully making the overall picture clearer on investment trade-off between hive expansion and reproduction.\\\\
    There were multiple ways to shift optimal reproduction time earlier.  Higher nutrition condition, increasing individual productivity and earlier flowering period were the three factors mentioned.  Inter-generational environment feedback have provided resilience to the plant-bee interaction couple.  The brief interaction was listed as follow:  Bees must maximize their hive gain from the flower or else cheaters may squeeze into this mutualism relationship.  While gaining benefits, over-exploitation of nectar can lead to next-generation resource depletion while the opposite may cause resource overflow.  Since resource overflow will strengthen the hive while the opposite will strike down a hive's health, some hives may active responding to the shift by not altering hive response to floral changes.  Given a higher ability adapting to floral activity variability, annual bees life cycles can only be affected by long-terms floral cycle shifts.\\\\
    In conclusion, climate change will force floral cycle to fluctuate more.  Yet bees will only be affected by long-term trend and phenological mismatches will only show its effect after generation lags.
    \clearpage
    
    \section{Advances in increasing the understanding of insect pollinators in Kenya}
    \begin{flushright}
        \large{Dr. Esther Kioko (National Museum of Kenya)}\\
        seminar date: 04-Dec-2019
    \end{flushright}
    Climate change has been affecting Kenya, an east African country.  Land use change, chemical pollution and extreme weather events have been identified as major threats to the regional native pollinators -- bees.  In this seminar, the speaker has introduced latest insights on how academics and locals can collaborate on bees protection.\\\\
    Dry-lands and semi-arid areas have covered more than 80\% land in the country.  Contemporary practices of fire-clearing strategy before re-cultivation has been proved effective on agriculture yet destructive to the native biodiversity.  Interestingly there are consensus between academics and locals in the aspect of bee protection via ``bee-keeping" approach.  Through questionnaires, researchers know that the proportion of farmers willing to actively participating in hosting a hive is dominant.  The government is also welcoming this idea to provide additional refuge for native bees and extra income for farmers.  Under this positive atmosphere towards bee-hosting, the chance of such policy being implemented is high.\\\\
    Bees were found having dominant impact to fruit-development after fertilization period.  Out of expectation, self-fertilization of peppers, cucumbers, spider plants and other commercial plants has observable effects.  Fruits from self-fertilization are found to be more irregular-shaped, weigh lighter and smaller than the bee-pollinated ones.  The result has been further providing evidence to the ecological and economic importance of native bees to the region.\\\\
    With initiatives and values of bee-protection being documented, the future of native African bees is bright.
    \clearpage
    
    \section{The complex consequences of simple sociality in the wild}
    \begin{flushright}
        \large{Dr. Josh Firth (University of Oxford)}\\
        seminar date: 05-Dec-2019
    \end{flushright}
    Social birds are generally in large flocks.  Hence social hierarchy and interactions might be an important factor to determine fitness of an individual.  This seminar have not only demonstrated the importance of social linkages between birds, but also gave insights to humans.\\\\
    Experiments using tag-identification feeders have provided solid data to philosophical answers.  Individuals are more likely to mate with another individual of the opposite sex within their reach.  Cross-patched mating pairs are more likely to stay in non-favourable grounds to themselves if their partners are benefited.  Strategies are developed within the pair to survive in a non-favourable ground.\\\\
    On top of that, these experiments have shown that male personal has positive impact on mate bond strength.  Bold males are likely to meet their partners early in lives while spending more time with their mates for the rest of their lives.  Results have also been shown that individuals within a social net are replaceable.  New replacements are more and stronger than the lost ones.\\\\
    Within a flock, introverts are found to have fewer but stronger bonds comparing to extroverts.  When facing challenges, problem-solvers are found to have opposite social status between juvenile and adult groups.  They are better linked with others when they are adults.  Extroverts were also observed to spread simple knowledge better than introverts but the opposite behaviour was simulated for complex behaviour.\\\\
    The above presentation is a good insight of speculations to our own online and physical social behaviour.  This also gives us data to convince ourselves if one wants to perform better in their social circles.
    \clearpage
    
    \section{Reconstructing the spread of bacterial mobile elements in space and time}
    \begin{flushright}
        \large{Prof. Francois Balloux (University College London)}\\
        seminar date: 09-Jan-2020
    \end{flushright}
    ``Genes should be the target of combat towards virulence and deadly disease."  This message summed up the speaker's seminar well but unfortunately this was, is and will not be a fact in foreseeable time.
    
    Although probability of having horizontal gene transfer (HGT) between bacterial cells is low, the vast number of bacterial cells made the HGT frequent.  Adding onto the existence of transposon elements, which their mobility source from the environment, granted high but decaying transmission ability to the element.  As bacteria can gain different types of ``armor" through different plasmids, these clonal organisms have variable genomes across individuals.  Hence the varying elements have given difficulties of identification under whole genome sequencing approach, giving rise to ``core genome" and ``accessory genome" under large number of cell samples.  The mechanisms of lineages gaining virulence and/or resistance genes were still under debate, yet these elements have given rise to powerful and tough pathogens under human past and present (inappropriate) medical interference.
    
    Fortunately, rapid evolution of virulence in pathogens left a window open in phylogenetics forensics.  The speaker has done a \textit{Klebsiella pneumoniae} phylogeny for an outbreak in a Chinese hospital, which proved inoocence of a patient blamed to be the source of the outbreak.  Because of the reference databases, the study not only can date the time of existence for the ancestral strain but also identify plasmid types, sizes and resistance and virulence genes they carried.  Through transmission mapping, the study provided an accurate indication of transmission chains and approximate time of spread.  Although this technique can only reconstruct past events, potential of analyzing and developing a prediction model on outbreak patterns, causes and gene combination requirements is probably worth investigating.
    
    Yet all the above can probably be constrained within the academics.  The reality in clinical situation is a totally different world.
    \clearpage
    
    \section{Managing fisheries to protect dependant predators}
    \begin{flushright}
        \large{Dr. Simeon Hill (University of Cambridge)}\\
        seminar date: 16-Jan-2020
    \end{flushright}
    The most popular and convenient way of managing fisheries is ``single-species management", which aims at limiting harvest rate to maintain a sustainable population for future human and predator consumption.  It was effective for all the years despite the fact that global fish stock is cascading.  Southern Ocean krill management has a long history and has been very conservative.  Hence the industry and predators are coexisting well, as reflected by models and predator reference points (i.e. predator census in the field).
    
    However, good examples are only rare cases.  Among the six commercial prey fishes under investigations, krill is the only exceptional cases which predators population can rise.  Predators for the other five are suffering different levels of population drop.  Even worse, data from predator population is usually available for the fisheries managers but it is often unused.  The phenomena showed the ignorance of managers to the responsibilities they have to carry.
    
    Public awareness is another aspect discussed in the seminar, which the result is exceptional.  Many interviewees aware and know current measures and their effects on the nature but they are unsure about what problems these measures aimed at.  Also, the result was a binary split between general public and companies (with scientists community being about 1:1 split between these two groups) on the topic of ``protection before exploitation".  Apparently public opinion is inclined towards protecting the environment before making benefit out of it while companies prefer otherwise.  However the discussion did not consider the source of profit for the companies, which definitely is from the public.  If we take this factor into account, the discussion would be the general public putting pressure and responsibilities on companies to put the environment first.  Yet if the companies don't care, the public also do not care; so the companies can still make profit and continue carrying on their ``exploitation first" business mode.
    
    Apart from the societal aspect, the political side is also logically cunning.  ``Strategic ambiguity" is a usual method to make a deal between stakeholders even though the content maybe vague.  The original idea was good as it pulled parties together to address a common issue.  However practically the deal was able to be interpreted in different way and hence in the real situation, the commercial fishes' predator populations are declining although multiple agreements were done in 1982, 1995 and 2010 respectively.
    \clearpage
    
    \section{Effects of Temperature on Microbial Metabolic Rates: Linking of Individual Responses to Ecosystem Impacts}
    \begin{flushright}
        \large{Dr. Tom Smith (Imperial College London)}\\
        seminar date: DD-MMM-2020
    \end{flushright}
    Although eubacteria and archaea have a higher average than global (0.65), community composition under different temperatures can be very different even the starting community is the same.  Distribution of these lineages is right-skewed (i.e. some thermal-tolerant lineages) with two different trends.  Mesophiles are living in many different contemporary environment and they follow ``hotter is better" short-term temperature performance trend while thermal extremophiles follow ``equalisation" trend.  Microbial thermal sensitivity of respiration is also found higher in dense microbe environments.
    
    To know how to climate change may impact microbial community and potentially shift the equilibria of higher order ecosystems that depend on them (either for food or chemical cues), the presenter has mentioned microbial cultivation experiments under different temperatures.  The result showed short-term thermal shock do not shift equilibrium position of the microbial community, although apparent thermal sorting effect was huge (i.e. restorable composition, environmental change independent from evolution).  With 200 generation of \textit{E. coli}, the community has already adapted to a temperature increase of +7$o$C.  This demonstrated the stability of microbial community and adaptability.
    
    The seminar also showed the difficulty of parametrize observed phenomena on microbial community thanks to the complexity of microbial interactions and lack of thorough investigation of microbial strains.  Yet, studies have identified different life strategies among microbes.  This is potentially important, as K-specialists have more prolonged cascade influence to downstream macro-ecosystems.
    
    This seminar has successfully demonstrated the potential importance and need of investigation on thermal features of microbe lineages and as a community.  Although microbial compositional changes are usually elastic, the pendulum effect on macro-ecosystem is probably alarming.
    \clearpage
    
    \section{Ecosystem services in a changing world}
    \begin{flushright}
        \large{Coline Jaworski (Aix-Marseille University)}\\
        seminar date: 28-Jan-2020
    \end{flushright}
    Insect population is declining because of multiple factors.  Yet feedback mechanisms between pollinators and angiosperms are yet to be known well.  In this seminar, the presenter demonstrated how drought (a kind of climate change effect) affect floral phenotypic traits and hence impacting pollinators' food source, and how the population of pollinators having a reverse impact on the floral propagation towards the next growing season.
    
    Three major types of geographical and temporal effects were described under the influence of climate change and anthropogenic impacts - phenological mismatch (between floral fertile and peak pollinator activity period), floral resource production irregularity (increase frequency of mismatch between nectar production peak and pollinator visit peak periods) and lineage distribution shift (environmental stresses may alter habitable range between floral and pollinator lineages differently, altering their habitat overlapping area).  Using drought as a model impact, the presenter is investigating how this single shift in precipitation induce a chain effect on floral source and pollinator bees across time.
    
    Drought was found affecting significantly the olfactory traits of plants, which was long described as an important factor for insect-pollinated plants.  However, physical impacts on materials production in floral bodies were not as large as expected.  In drought years nectar production decreases and more fruits are produced.  Yet number of seeds and number of seeds per fruit was not found impacted by drought.
    
    Since nectar production was negatively influenced, the presenter speculated that it should lead to a downstream impact on the available energy budget in bee colonies depending on these flowers' nectar.  As expected, the presenter proposed behavioural experiments by setting up treatments of habitat enhancements.  More work will be done before the presenter can address her primary aim - how to mitigate contemporary insect population decline and manage a sustainable ecosystem service.  A mechanistic understanding of plant responses to climate change will be also uncovered along the progress.
    \clearpage
    
    \section{The phylogenetic signature of interspecific competition in birds}
    \begin{flushright}
        \large{Dr. Jonathan Drury (Durham University)}\\
        seminar date: 30-Jan-2020
    \end{flushright}
    Species interactions would lead to response-feedback behavioural interactions, yet this factor was seldom being investigated in phylogenetic context.  In this seminar, the speaker has used three researches as example to demonstrate some phylogenetic traceable behavioural interactions between species.
    
    The projects were layered into three scales - local, continental and altitudinal.  On a local level, interspecies interactions are more intense when they are more similar, speeding up diversification in other phenotypic traits.  Phylogenetic linear mixed models have discovered higher character displacement events occur when interacting species have high syntopy, similar songs or similar plumage.  Interspecific territoriality probability increases among lineages with breeding habitat overlap for non-hybridizing species pairs, while this behaviour is fairly high and stable between hybridization pairs.  In local perspective, the territoriality interactions match the ``ancestral resemblance" hypothesis.
    
    In the continental scale, ancestral biogeography is important when investigting phylogenetic signatures.  Overall speaking, the analyses need to address inter-relatedness between species.  Under the new method, sympatric features were found repelling each other in trait space.  Interspecific competition would leave stronger signals on long-term features such as resource usage and some plumage traits.  As social interaction traits evolve much faster and dynamic, it does not leave trace in phylogeny.
    
    Along latitudinal gradients, it has been long known that biotic pressure dominates selection in tropics while abiotic selection dominates the temperate regions.  However mechanisms resulting the above observations are not well-studied.  Under analyses on family level, niche saturation was found to restrict divergence.  Due to more tropical biomes are having more biodiversity, it is hence easier to be niche saturated and restricting divergence.  On continental-level analyses, however, the claim was only partly true.  Tropical restriction on divergence is true on high trophic level organisms oropen habitats.  However tropics and temperate regions were not differing in dense habitats and semi-open habitats.  The opposite was even found for low trophic levels, indicating apparent observations might not always suitable to be taken as the rule of thumb.
    
    In conclusion, phylogeny is a chronological record of evolution not only in phenotypical traits but also interactions within the distribution of the lineage between other species.  Hence deciphering the resultant genome after interactions across millions of years need to be careful and holistic.  Advanced methods should be further elaborated and extended.
    \clearpage
    
    \section{Predicting global biodiversity with global mechanistic models and neutral theory}
    \begin{flushright}
        \large{Lucas D. Fernandes (Imperial College London)}\\
        seminar date: 04-Feb-2020
    \end{flushright}
    Mechanistic models are useful to understand underlying pivots and gears determining our observations in nature.  However, formulating these description equations often fall into a trade-off between model simplicity versus realistic.  Neutral theory and Madingly model are the representatives of ``simple" and ``realistic" models respectively.
    
    Madingly model divides Earth surface into grids and heterotrophic life into 19 functional groups according to their realms, feeding modes, reproductive strategies and thermoregulation methods.  Using data of environment, oceanic currents and net primary productivity, it can spatially simulate biomass movements across the globe and density of biomass at specific location at each time step.  From the result of simulation, there is a periodic North-South shifting motion across continents.  However this result was only accurate for a few locations.  The reason of this model getting wrong was suspected to be the loss of species identity, which was an assumption for the model in global scale.
    
    On the other hand, neutral model covers the species identity aspect.  As an alternative mechanistic model, it covers aspects of reproduction from neighbouring individuals, speciation, dispersal and extirpation.  Result mostly would be a dynamical equilibrium of species richness, with reproducible data patterns from relative species abundances and species-area relationship.  Although being the good null model in macroecology, neutral model has its disadvantage of being too detail.  Species identity is assigned to individual, which is unrealistic.
    
    Hence the presenter proposed a method of model coupling.  By combining Madingly and neutral models, each one's disadvantages are better compensated by the other's advantage.  Hence the result from the compound model would contain a comparatively holistic view with relatively precise player identities.  In this new model, they used body size as identity guides and speciation rate as the free parameter.  Due to the high discrepancy between speciation rate and average organism longevity, the accurate body mass guide is insufficient to bring up explanatory power alone.  Hence several hypotheses based on the model have been raised, including model optimizations, parameter optimizations and developing new methods to estimate parameters.
    
    In conclusion, the presenter agreed that models are always wrong, but a good model should be wrong in ways that can provide insights to the underlying biological mechanisms. Hence understanding the reason(s) leading to discrepancies between model results and observations is scientifically important.

\end{document}
